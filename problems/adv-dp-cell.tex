\begin{problem}{일하는 세포}
    {표준 입력}{표준 출력}
    {1 초}{512 MB}{}
    
    사람의 세포 수, 약 37조 개. 세포들은 몸 속에서 오늘도 열심히 일하고 있다. 그중에서도 우리의 적혈구는 혈관을 타고 돌아다니며 산소와 영양소를 운반해주는 중요한 역할을 맡고있다.
    
    적혈구는 심장이나 폐 같은 거점들을 돌아다니면서 산소와 영양분을 운반한다. 몸 속에는 총 $ N $개의 거점이 있고, 몇몇 거점은 통로를 통해 서로 이어져 있다. 거점 사이의 통로를 통과하는데는 1초가 걸린다. 하지만 혈관의 곳곳에는 판막이나 공사중인 부분들이 있기 때문에 매 초 거점 사이의 연결관계가 바뀐다. 그럼에도 불구하고 몸의 곳곳이 산소와 영양분을 필요로 하므로 적혈구는 가만히 있을 수 없으며, 매 초 통로를 무조건 하나 타야 한다. 일부 통로는 출발 거점과 도착 거점이 같을 수도 있다. 일부 거점의 특정 순간에는 나가는 통로가 없을 수도 있는데, 이 때는 도착한지 1초 후에 파괴되어 몸과 다시 하나가 된다. 잔혹하지만 우리의 몸은 이렇게 돌아간다.
    
    우리의 적혈구는 매 순간 변하는 몸속 혈관 지도에 길을 헤매지만 그래도 최선을 다해서 하루하루 열심히 일을 하고 있다. 옆에 있던 백혈구가 길을 헤매는 적혈구를 보고 도와주고 싶다는 생각을 했다.
    
    수십 시간의 유주 경험을 통해 백혈구는 몸속 혈관 지도가 $ T $초를 주기로 반복된다는 것을 알게 되었다. 이 사실을 정리해서 적혈구가 거점 $ A $에서 출발하여 정확히 $ D $초 후 거점 $ B $에 도달하게 되는 경우의 수를 모든 거점의 순서쌍에 대해 구해주고자 하지만 너무나도 단세포이기 때문에 머리가 나빠서 계산을 하지 못했다. 한 경로는, $ D $초 동안 통과한 통로의 순열로 정의된다. 백혈구를 도와서 적혈구가 $ D $초 동안 한 거점에서 다른 거점까지 움직일 수 있는 경우의 수를 구해주자!
    
    \InputFile
    
    첫 번째 줄에는 백혈구가 알아낸 혈관 지도들의 주기인 자연수 $ T $와 거점의 개수인 자연수 $ N $, 적혈구가 움직이는 시간인 정수 $ D $가 공백으로 구분되어 주어진다. ($ 1 \leq T \leq 100 $, $ 2 \leq N \leq 20 $, $ 0 \leq D \leq 10^9 $)
    
    그 뒤 거점 사이의 연결 관계를 나타내는 혈관 지도 $ T $개가 순서대로 1번부터 $ T $번까지 주어지는데, 혈관 지도가 주어지는 형식은 다음과 같다.
    
    \begin{itemize}
        \item 첫 번째 줄에는 거점 사이를 잇는 혈관의 개수인 자연수 $ M_i $ 가 주어진다. ($ 0 \leq M_i \leq N^2 $)
        \item 그 뒤 $ M_i $ 개의 줄에 걸쳐 세 자연수 $ a $, $ b $, $ c $가 공백으로 구분되어 주어진다. 이는 거점 $ a $에서 거점 $ b $로 가는 서로 다른 단방향 통로가 $ c $ 개 있음을 의미한다. ($ 1 \leq a,\ b \leq N $, $ 1 \leq c \leq 1000 $)
    \end{itemize}
    매 혈관 지도에 중복된 연결 관계는 주어지지 않으며, $ i $초에서 $ (i+1) $초 동안 이동할 때는 $ (i\ \%\ T + 1) $번 혈관 지도가 적용된다. $ i\ \%\ T $는 $ i $를 $ T $로 나눈 나머지를 의미한다.
    
    \OutputFile
    출력은 $ N $개의 줄로 구성되며, $ i $번째 줄에는 $ N $개의 정수 $ x_{i1},\  x_{i2},\ \cdots,\ x_{iN} $를 공백으로 구분하여 출력해야 한다. $ x_{ij} $는 0초 때 거점 $ i $에서 출발하여 정확히 $ D $초 때 거점 $ j $에 위치하게 되는 경로의 수를 1,000,000,007로 나눈 나머지이다.
    
    \Examples
    
    \begin{example}
        \exmp{
            1 2 4
            2
            1 1 2
            2 2 3
        }{%
            16 0
            0 81
        }%
    \exmp{
            3 4 5
            4
            1 2 1
            2 3 1
            3 4 1
            4 1 1
            4
            1 1 2
            2 3 1
            3 4 1
            4 2 1
            5
            1 2 1
            2 4 1
            4 3 1
            3 1 1
            1 4 1
        }{%
            0 0 1 0
            0 1 0 0
            2 0 0 0
            4 0 0 2
        }%
    \exmp{
            1 8 3
            12
            1 2 1
            1 8 1
            2 3 1
            2 8 1
            3 4 1
            3 7 1
            3 8 1
            4 5 1
            4 7 1
            5 6 1
            6 7 1
            7 8 1
        }{%
            0 0 0 1 0 0 1 1
            0 0 0 0 1 0 1 1
            0 0 0 0 0 1 0 1
            0 0 0 0 0 0 1 0
            0 0 0 0 0 0 0 1
            0 0 0 0 0 0 0 0
            0 0 0 0 0 0 0 0
            0 0 0 0 0 0 0 0
        }%
    \end{example}
    
    \begin{example}
        \exmp{
            1 8 100000000
            24
            1 2 1
            1 8 1
            2 1 1
            2 3 1
            2 8 1
            3 2 1
            3 4 1
            3 7 1
            3 8 1
            4 3 1
            4 5 1
            4 7 1
            5 4 1
            5 6 1
            6 5 1
            6 7 1
            7 3 1
            7 4 1
            7 6 1
            7 8 1
            8 1 1
            8 2 1
            8 3 1
            8 7 1
        }{%
            261245548 769313318 167840464 862450688 445583828 270525651 828293276 976953408
            769313318 542054741 65223362 957807341 63263610 545566670 134857214 863984679
            167840464 65223362 959076197 916983285 988077461 199284746 461375786 371787307
            862450688 957807341 916983285 119640157 267995930 978327505 847171719 483910227
            445583828 63263610 988077461 267995930 394594439 718634258 715295769 69712722
            270525651 545566670 199284746 978327505 718634258 131562703 197248645 728310434
            828293276 134857214 461375786 847171719 715295769 197248645 322189612 142912983
            976953408 863984679 371787307 483910227 69712722 728310434 142912983 162707920
        }%
    \end{example}
    
    \Explanation
    
    네 번째 예제의 경우 적혈구가 3.17년 동안 몸 속을 돌아다니는 경우이다. 표의 너비가 좁아 8개의 숫자들이 한 줄에 있어야 하나 3개 / 3개 / 2개로 나뉘어 표시되었다.
    
\end{problem}

