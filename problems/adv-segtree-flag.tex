\begin{problem}{깃발춤}
    {표준 입력}{표준 출력}
    {2 초}{512 MB}{}
    
    매일 코딩만 하던 상헌이는 두뇌가 너무 코드에만 범벅이 되고 있는 것 같아 두뇌를 다른 방면으로 몰입시킬 수 있는 깃발춤 공연을 보러 가기로 했다. 깃발춤 공연은 $ N $명의 공연자들이 일렬로 서서 깃발을 박력 있게 흔들며 진행된다. 깃발을 들고 있는 공연자들은 각각 카리스마의 정도 $ c_i $를 지니고 있어, 몇몇 공연자들은 보다 절도 있게 깃발을 흔든다.
    
    상헌이는 깃발춤을 보다가, 문득 깃발춤 공연 중 몇몇 연속된 공연자들이 깃발을 교대로 흔드는 것을 목격했다. 상헌이는 이를 `교대 깃발춤'이라 명명하였다. 교대 깃발춤은 $ L $번째 공연자부터 $ R $번째 공연자까지 깃발을 각자 왼쪽 또는 오른쪽으로 흔드는 동작인데, $ L $번째 공연자를 포함하여 $ L $번째 공연자과의 거리가 짝수인 공연자는 깃발을 왼쪽으로 흔들고, 거리가 홀수인 공연자는 오른쪽으로 흔든 뒤, 다시 깃발을 몸 쪽으로 원위치시킨다. 여기서 $ x $번째 공연자와 $ y $번째 공연자 사이의 거리는 $ |x - y| $로 표현된다.
    
    문제 해결에서 벗어날 수 없었던 상헌이는 교대 깃발춤에서 왼쪽으로 깃발을 흔든 공연자들의 카리스마의 합과 오른쪽으로 깃발을 흔든 공연자들의 카리스마의 합의 차이의 절댓값을 교대 깃발춤의 균일도라고 부르기로 하였다. 교대 깃발춤의 균일도가 큰 값을 가지면, 한쪽이 다른 쪽보다 압도적으로 카리스마가 느껴진다는 뜻이기에 비대칭적으로 보일 수 있다. 상헌이는 교대 깃발춤의 균일도가 중요한 의미를 지닌다고 생각한다. 또 깃발을 흔드는 공연자들은 공연의 열기와 순간순간의 실수에 휩쓸리기 때문에, 공연자들의 카리스마가 증가하거나 감소할 수 있다. 이런 모든 상황을 고려하며 상헌이는 매 교대 깃발춤의 균일도를 구하고 싶어졌다. 상헌이를 도와주자!
    
    \InputFile
    
    첫 번째 줄에는 깃발춤을 진행하는 공연자의 명수인 자연수 $ N $과 상황 변화의 개수인 자연수 $ Q $가 공백으로 구분되어 주어진다. ($ 1 \leq N \leq 300,000 $, $ 1 \leq Q \leq 300,000 $)
    
    두 번째 줄에는 정수 $ c_1,\ c_2,\ \cdots,\ c_N $ 이 공백으로 구분되어 주어지며, $ c_i $ 는 $ i $번째 공연자의 카리스마를 의미한다. ($ -100,000 \leq c_i \leq 100,000 $)
    
    세 번째 줄부터 $ Q $개의 줄에 걸쳐 다음 형식 중 하나로 세 정수가 공백으로 구분되어 주어진다.
    \begin{itemize}
        \item \verb|1| $ L $ $ R $ : $ L $번째 공연자부터 $ R $번째 공연자까지 구성된 교대 깃발춤이 시연된다. ($ 1 \leq L \leq R \leq N $)
        \item \verb|2| $ L $ $ x $ : $ L $ 번째 공연자의 카리스마가 정수 $ x $만큼 증가한다. ($ 1 \leq L \leq N $, $ -100,000 \leq x \leq 100,000 $)

    \end{itemize}
    첫 번째 종류(`\verb|1| $ L $ $ R $' 꼴)의 쿼리는 한 번 이상 주어짐이 보장된다.
    
    \OutputFile
    첫 번째 종류의 쿼리가 입력될 때마다 매 줄에 해당하는 교대 깃발춤의 균일도를 출력한다.
    
    \Examples
    
    \begin{example}
        \exmp{
            6 3
            3 1 4 1 5 9
            1 2 4
            2 3 10
            1 3 6
        }{%
            2
            9
        }%
    \end{example}
    
    \Explanation
    
    첫 번째 쿼리에서 교대 깃발춤을 추는 공연자들의 카리스마는 순서대로 1, 4, 1이다. 균일도는 $ |(1 + 1) - (4)| $ = 2이다.
    
    두 번째 쿼리로 인해 3번째 공연자의 카리스마가 10 증가하여, 전체 카리스마가 3, 1, 14, 1, 5, 9가 된다.
    
    세 번째 쿼리에서 교대 깃발춤을 추는 공연자들의 카리스마는 순서대로 14, 1, 5, 9이다. 균일도는 $ |(14 + 5) - (1 + 9)| $ = 9이다.
    
\end{problem}

