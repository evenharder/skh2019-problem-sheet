\begin{problem}{핑거 스냅}
    {표준 입력}{표준 출력}
    {2 초}{512 MB}{}
    
    [어벤져스] 시리즈를 보지 않은 사람이라도 `인피니티 건틀렛'이 무엇인지는 다들 알 것이다. 그래도 모르는 사람들을 위해 설명을 하자면, 인피니티 스톤이 모두 모인 인피니티 건틀렛을 끼고 손가락을 튕기면, 사용자가 원하는 것을 할 수 있다. 그러나 반동이 매우 심하기 때문에 그리 많이는 사용할 수 없다.
    
    정신 나간 수학자 Sonaht는 우연히 이 인피니티 건틀렛을 손에 넣게 된다. 그러나 이 인피니티 건틀렛에는 약간의 하자가 있어서, 핑거 스냅으로 할 수 있는 일이 몇가지 없다. 다음은, 핑거 스냅으로 할 수 있는 일을 나열한 것이다.
    
    \begin{enumerate}
        \item 전 우주의 생명체 수를 현재의 절반으로 한다.
        \item 전 우주의 생명체 수를 현재의 1/3로 한다.
            
        \item 전 우주의 생명체 수를 현재보다 하나 늘린다.
        \item 전 우주의 생명체 수를 현재보다 하나 줄인다. 이미 전 우주의 생명체 수가 0이라면 할 수 없다.
    \end{enumerate}
    첫 두 경우에서, 나누어 떨어지지 않으면 몫만 남기고, 나머지는 버린다.
    
    Sonaht는 전 우주의 생명체 수를 목표치 $ A $ 이상 $ B $ 이하로 만들려고 한다. 그러나 역시나 정신 나간 수학자답게, $ A $ 이상 $ B $ 이하인 수 중 소수로 만들려 한다 (\textbf{어쩌면 $ A $와 $ B $ 사이에 소수가 없을지도 모르지만 말이다.}) 소수란, 서로 다른 약수가 1과 자기 자신밖에 없는 수를 의미한다. 그러나 인피니티 건틀렛은 반동이 심하기에, Sonaht는 최대한 적은 수의 핑거 스냅으로 이 목표를 달성하고자 한다. Sonaht가 최소 몇 번의 핑거 스냅을 해야 할지 구해보자.
    
    \InputFile
    첫 번째 줄에 테스트 케이스의 개수 $ T $가 주어진다. ($ 1 \leq T \leq 10 $)
    
    두 번째 줄부터 $ T $개의 줄에 걸쳐, 현재 전 우주의 생명체 수인 자연수 $ N $과, Sonaht의 목표 범위인 자연수 $ A $, $ B $가 공백으로 구분되어 주어진다. ($ 2 \leq N \leq 1,000,000,\ 2 \leq A \leq B \leq 100,000 $)
    
    \OutputFile
    매 줄마다, 각 테스트 케이스에서 Sonaht가 전 우주의 생명체 수를 목표범위 내의 소수로 만드는 데 필요한 최소한의 핑거 스냅의 횟수를 출력한다.
    
    만약 목표범위 내의 소수로 만들 수 없다면, \verb|-1|을 출력한다.
    
    매 테스트 케이스는 독립적으로 고려되어야 한다.
    
    \Examples
    
    \begin{example}
        \exmp{
            5
            9 2 4
            100000 605 610
            300 14 16
            7 7 10
            98765 500 550
        }{%
            1
            15
            -1
            0
            12
        }%
    \end{example}
    
    \Explanation
    첫 번째 테스트 케이스에서는, 2번 동작 (전 우주의 생명체를 1/3로 줄임)을 하면 전 우주의 생명체의 수가 3이 되어, 2 이상 4 이하 소수가 된다. 그러므로 필요 횟수는 1이다.
    
    세 번째 테스트 케이스에서는, 14 이상 16 이하 소수가 존재하지 않으므로, 아무리 핑거 스냅을 해도 목표를 달성할 수 없다. 따라서 \verb|-1|을 출력한다.
    
    네 번째 테스트 케이스에서는 핑거 스냅을 하지 않고도 목표가 달성되으므로 필요 횟수는 0이다.
    
\end{problem}

