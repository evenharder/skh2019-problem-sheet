\begin{problem}{이건 꼭 풀어야 해!}
    {표준 입력}{표준 출력}
    {1 초}{512 MB}{}
    
    숭실골 높은 언덕 깊은 골짜기에 출제로 고통 받는 욱제가 살고 있다!
    
    욱제는 또 출제를 해야 해서 단단히 화가 났다. 그래서 욱제는 길이 $ N $짜리 수열 $ A $를 만들고, $ A $를 비내림차순으로 정렬해서 수열 $ B $를 만들어 버렸다!! 여기서 $ B $를 출력하기만 하면 문제가 너무 쉬우니까 하나만 더 하자. 아래와 같은 질문이 무려 $ Q $개나 주어진다!! (ㅎㅎ;; ㅈㅅ.. ㅋㅋ!!)
    \begin{itemize}
    \item \textbf{L} \textbf{R}: $ B_L + B_{L+1} + \cdots + B_{R-1} + B_R $ 을 출력한다.
    \end{itemize}
    \begin{figure}[h]
        \centering
        \includegraphics[width=0.15\textwidth]{blobaww.png}
        \caption{모든 참가자가 문제를 풀 수 있을 것이라고 기대하는 욱제의 표정}
    \end{figure}
    욱제의 질문에 답하고 함께 엠티를 떠나자!!
    
    \InputFile
    첫 번째 줄에 수열 $ A $의 길이 $ N $과 질문의 개수 $ Q $가 공백으로 구분되어 주어진다. ($ 1 \leq N,\ Q \leq 300,000 $)
    
    두 번째 줄에 $ N $개의 정수 $ A_1,\ A_2,\ \cdots,\ A_N $ 이 공백으로 구분되어 주어진다. $ A_i $는 수열 $ A $의 $ i $번째 수이다. ($ 1 \leq A_i \leq 1,000 $)
    
    세 번째 줄부터 $ Q $개의 줄에 걸쳐 욱제의 질문을 의미하는 두 수 $ L $, $ R $이 공백으로 구분되어 주어진다. ($ 1 \leq L \leq R \leq N $)
    
    주어지는 모든 입력은 자연수이다.
    
    \OutputFile
    $ Q $개의 줄에 걸쳐, 질문의 답을 순서대로 각각 출력한다.
    
    \Examples
    
    \begin{example}
        \exmp{
            5 6
            2 5 1 4 3
            1 5
            2 4
            3 3
            1 3
            2 5
            4 5
        }{%
            15
            9
            3
            6
            14
            9
        }%
        \exmp{
            5 3
            2 5 1 2 3
            1 3
            2 3
            1 5
        }{%
            5
            4
            13
        }%
    \end{example}
    
    \Explanation
    첫 번째 예제에서 \verb|[2, 5, 1, 4, 3]|을 비내림차순으로 정렬하면 \verb|[1, 2, 3, 4, 5]|이다.
    
    두 번째 예제에서 \verb|[2, 5, 1, 2, 3]|을 비내림차순으로 정렬하면 \verb|[1, 2, 2, 3, 5]|이다.
    
    \Note
    비내림차순은 원소가 감소하지 않는 (같거나 증가하는) 순서를 말한다.
    
    \begin{framed}
    \begin{verbatim}
while (Q--) {
    int sum = 0, L, R;
    scanf(“%d %d”, &L, &R);
    for (int i = L; i <= R; i++) {
        sum += a[i];
    }
    printf(“%d\n”, sum);
}\end{verbatim}
    \end{framed}
    위와 같이 각 질문마다 반복문을 매번 돌려서 답을 계산하면, 시간복잡도가 $ O(QN) $ 이 되므로 시간 초과를 받게 된다. 다른 방법을 이용해 문제를 해결해야 한다.
        
    
\end{problem}

