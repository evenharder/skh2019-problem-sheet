\begin{problem}{우울한 방학}
    {표준 입력}{표준 출력}
    {1 초}{512 MB}{}
    
    방학동안 기숙사에 홀로 남겨진 인호는 우울하고 고독하다. 다행히 인호는 $ M $일의 방학 동안 $ N $개의 약속이 잡혀있기에, 약속 날짜의 효율적인 배치를 통해 방학 내에 느낄 우울함의 합을 최소화하려고 한다.
    
    인호의 기분은 정수로 표현 가능하며, 기분이 0 미만인 날에 (기분)$ ^2 $ 만큼 우울함을 느낀다. 인호의 기분은 오늘 약속이 있다면 약속의 기대행복 값인 $ H_i $이며, 약속이 없으면 어제의 기분에서 1을 뺀 값이다.
    
    인호는 하루에 최대 한 개의 약속을 소화할 수 있으며, $ N $개의 약속들의 순서는 주어진 순서대로여야 한다.
    
    방학은 내일부터 시작이며, 오늘 인호의 기분은 0일 때, 약속을 적절히 배치하여 인호가 방학 동안 느낄 우울함의 합을 최소화하자.
    
    \InputFile
    첫 번째 줄에는 인호의 약속 개수인 자연수 $ N $과 방학의 일수인 자연수 $ M $이 공백으로 구분되어 주어진다. ($ 0 \leq N < M < 1000 $)
    
    두 번째 줄에는 $ N $개의 정수 $ H_1,\ H_2,\ \cdots,\ H_N $이 공백으로 구분되어 주어진다. $ H_i $는 $ i $번째 약속의 기대행복 값이다. ($ 1 \leq H_i < 100 $)
    
    \OutputFile
    첫 번째 줄에 인호가 방학 동안 느낄 우울함의 합의 최솟값을 출력한다.
    
    \Examples
    
    \begin{example}
        \exmp{
            3 10
            2 2 1
            }{%
            2
        }%
    \end{example}
    
    \Explanation
    1일, 5일, 8일에 약속을 순서대로 배치하면, 4일과 10일에 각각 1만큼의 우울함을 느끼게 되어, 총 2만큼의 우울함을 느끼게 된다. 이보다 덜 우울함을 느끼게 만드는 방법은 존재하지 않는다.
    
\end{problem}

