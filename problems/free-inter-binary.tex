\begin{problem}{이진수 변환}
    {표준 입력}{표준 출력}
    {1 초}{512 MB}{}
    
    당신에게 자연수 $ x_0 $와 $ N $이 주어졌다. 지금부터 당신은 이 자연수 $ x_0 $를 $ N $번의 `변환'을 통해 0으로 바꿀 것이다. 변환이란, 양의 정수를 이진법으로 표기했을 때, 1개 이상의 1을 0으로 바꾸는 작업이다. 예를 들어 9를 이진법으로 나타내면 $ 1001_{(2)} $인데, 9는 $ 0(\underline{0}00\underline{0}_{(2)}) $, $ 1(\underline{0}001_{(2)}) $, 또는 $ 8(100\underline{0}_{(2)}) $로 변환될 수 있다 (바뀐 자릿수는 밑줄로 표기되었다). 여러분의 목표는 $ x_i $를 변환하여 $ x_{i+1} $를 만드는 과정을 반복해, $ x_N $을 0으로 만드는 것이다.
    
    위 조건을 만족하는 수열 $ X = [x_0,\ x_1,\ x_2,\ \cdots,\ x_N ]$는 존재하지 않을 수도 있지만, 여러 개가 존재할 수도 있다. 만약 존재한다면, 각 수열별로 인접한 원소들의 차들의 집합 $ D(X) = \{x_0-x_1,\ x_1-x_2,\ \cdots, \ x_{N-1}-x_N\} $를 정의하자. 이 집합의 원소들의 최대값과 최소값의 차이를 최소화하도록, 수열 $ X $를 만들고자 한다. 즉, 가능한 모든 수열 $ X_i $ 중 ($ D(X_i) $에 속한 원소의 최댓값 $ - $ $ D(X_i) $에 속한 원소의 최솟값)이 최소가 되는 $ X_i $를 찾고자 한다.
    
    이상해보일 수 있는 문제지만, 당신은 대답해야 한다. 과연 1초 안에 답할 수 있을까?
    
    
    \InputFile
    
    첫 번째 줄에 변환할 자연수와 변환 횟수를 의미하는 두 자연수 $ x_0 $과 $ N $이 공백으로 구분되어 주어진다. ($ 1 \leq x_0 \leq 10^{16} , 1 \leq N \leq 50 $)
    
    \OutputFile
    만약 조건을 만족하는 수열이 존재하지 않으면 첫 번째 줄에 \verb|-1|을 출력한다.
    
    조건을 만족하는 수열이 존재한다면, 수열의 원소를 의미하는 $ N $개의 정수 $ x_1,\ x_2,\ \cdots,\ x_N $을 공백으로 구분하여 출력한다.
    
    조건을 만족하는 수열이 여러 개 존재한다면, 아무 것이나 출력해도 좋다.
    
    \Examples
    
    \begin{example}
        \exmp{
            23 2
        }{%
            16 0
        }%
        \exmp{
            48 5
        }{%
            -1
        }%
    \end{example}
    
    \Explanation
    
    첫 번째 예제에서 $ x_0 = 23 = 10111_{(2)} $, $ x_1 = 16 = 10000_{(2)} $, $ x_2 = 0 = 00000_{(2)} $ 로 수열을 구성하면 인접한 원소의 차의 최대값과 최소값의 차이는 $ 16 - 7 = 9 $가 되며, 이보다 작게 만들 수 있는 방법은 존재하지 않는다. 또 다른 답으로는 $ x_0 = 23 $, $ x_1 = 7 $, $ x_2 = 0 $이 있다.
    
    두 번째 예제에서 주어진 조건을 만족하는 $ x_1,\ x_2,\ x_3,\ x_4,\ x_5 $는 존재하지 않는다.
    
\end{problem}

