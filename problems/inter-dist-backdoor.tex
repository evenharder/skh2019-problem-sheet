\begin{problem}{백도어}
    {표준 입력}{표준 출력}
    {2 초}{512 MB}{}
    
    유섭이는 무척이나 게으르다. 오늘도 할 일을 모두 미뤄둔 채 열심히 롤을 하던 유섭이는 오늘까지 문제를 내야 한다는 사실을 깨달았다. 그러나 게임은 시작되었고 지는 걸 무척이나 싫어하는 유섭이는 어쩔 수 없이 백도어를 해 게임을 최대한 빠르게 끝내기로 결심하였다.
    
    최대한 빨리 게임을 끝내고 문제를 출제해야 하기 때문에 유섭이는 최대한 빨리 넥서스가 있는 곳으로 달려가려고 한다. 유섭이의 챔피언은 총 $ N $개의 분기점에 위치할 수 있다. 0번째 분기점은 현재 유섭이의 챔피언이 있는 곳을, $ N-1 $번째 분기점은 상대편 넥서스를 의미하며 나머지 $ 1,\ 2,\ \cdots,\ N-2 $번째 분기점은 중간 거점들이다. 그러나 유섭이의 챔피언이 모든 분기점을 지나칠 수 있는 것은 아니다. 백도어의 핵심은 안 들키고 살금살금 가는 것이기 때문에 적 챔피언 혹은 적 와드(시야를 밝혀주는 토템), 미니언, 포탑 등 상대의 시야에 걸리는 곳은 지나칠 수 없다.
    
    입력으로 각 분기점을 지나칠 수 있는지에 대한 여부와 각 분기점에서 다른 분기점으로 가는데 걸리는 시간이 주어졌을 때, 유섭이가 현재 위치에서 넥서스까지 갈 수 있는 최소 시간을 구하여라.
    
    \InputFile
    
    첫 번째 줄에 분기점의 수와 분기점들을 잇는 길의 수를 의미하는 두 자연수 $ N $과 $ M $이 공백으로 구분되어 주어진다. ($ 2 \leq N \leq 100,000,\ 1 \leq M \leq 300,000 $)
    
    두 번째 줄에 각 분기점이 적의 시야에 보이는지를 의미하는 $ N $개의 정수 $ a_0,\ a_1,\ \cdots,\ a_{N-1} $가 공백으로 구분되어 주어진다. $ a_i $가 0이면 $ i $번째 분기점이 상대의 시야에 보이지 않는다는 뜻이며, 1이면 보인다는 뜻이다. 추가적으로 $ a_0 = 0 $, $ a_{N-1} = 1 $이다. $ N-1 $번째 분기점은 상대 넥서스이기 때문에 어쩔 수 없이 상대의 시야에 보이게 되며, 또 유일하게 상대 시야에 보이면서 갈 수 있는 곳이다.
    
    다음 $ M $개의 줄에 걸쳐 세 정수 $ a,\ b,\ t $가 공백으로 구분되어 주어진다. ($ 0 \leq a, b < N$, $a \neq b$, $1 \leq t \leq 100,000 $) 이는 $ a $번째 분기점과 $ b $번째 분기점 사이를 지나는데 $ t $만큼의 시간이 걸리는 것을 의미한다. 연결은 양방향이며, 한 분기점에서 다른 분기점으로 가는 간선은 최대 1개 존재한다.
    
    \OutputFile
    첫 번째 줄에 유섭이의 챔피언이 상대 넥서스까지 안 들키고 가는데 걸리는 최소 시간을 출력하여라. 만약 상대 넥서스까지 갈 수 없으면 \verb|-1|을 출력하여라.
    
    \Examples
    
    \begin{example}
        \exmp{
            5 7
            0 0 0 1 1
            0 1 7
            0 2 2
            1 2 4
            1 3 3
            1 4 6
            2 3 2
            3 4 1
        }{%
            12
        }%
        \exmp{
            5 7
            0 1 0 1 1
            0 1 7
            0 2 2
            1 2 4
            1 3 3
            1 4 6
            2 3 2
            3 4 1
        }{%
            -1
        }%
    \end{example}
    
    \Explanation
    
    첫 번째 예제에서 위 그래프의 최단거리는 0 \textrightarrow\ 2 \textrightarrow\ 3 \textrightarrow\ 4 를 지나는 시간인 5(2+2+1) 이지만, 3번 분기점이 상대의 시야에 있기 때문에 0 \textrightarrow\ 2 \textrightarrow\ 1 \textrightarrow\ 4를 지나는 시간인 12(2+4+6)이 최소 시간이 된다.
    
\end{problem}

