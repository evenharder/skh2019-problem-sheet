% !TeX document-id = {afd18b85-514b-4ec0-8413-70b08eddfedd}
%!TeX TXS-program:compile = txs:///xelatex/[--shell-escape]
\documentclass[11pt,a4paper,oneside,korean]{article}

\usepackage{xetexko}
\usepackage[utf8]{inputenc}
\usepackage[english]{babel}
\usepackage{olymp}
\usepackage{graphicx}
\usepackage{amsmath}
\usepackage{amssymb} 
\usepackage{textcomp}
\usepackage{framed}
\usepackage{color} % for colored text
\usepackage{import} % for changing current dir
\usepackage{epigraph}
\usepackage{daytime} % for displaying version number and date
\usepackage{wrapfig} % for having text alongside pictures
\usepackage{verbatim}
\usepackage{array} % compatibility check
\usepackage{multirow}
\usepackage{enumitem} % for item environment line spacing
\usepackage{skh2019}
%\usepackage{hangulfontset}

\graphicspath{{./images/}}
\renewcommand{\baselinestretch}{1.4}
\contest
{2019 숭고한 연합 Algorithm Camp Contest}%
{숭실대학교}%
{2019.08.09}%

\newcommand*{\NoInputFileName}{}
\newcommand*{\NoOutputFileName}{}

\binoppenalty=10000
\relpenalty=10000
\exhyphenpenalty=10000

%\setmainfont{Noto Sans}
%\setmainhangulfont{Noto Sans CJK KR}
%\setmonofont{D2Coding}

\begin{document}
    %\raggedbottom
    %\displayauthorinfootertrue
    %\intentionallyblankpagestrue
    \begin{figure}[h]
        \centering
        \includegraphics[height=0.12\textheight]{./logo-cropped.png}
    \end{figure}
    
    문제 배치는 다음과 같습니다.
    {
    \begin{table}[h]
        \centering
        \sffamily\large
        \renewcommand{\arraystretch}{1.2}
            \begin{tabular}{c|l|c|c|c}
            번호 & 문제명 & 초 & 중 & 고 \\
            A & 와글와글 숭고한 & O & & \\
            B & 보너스 점수 & O & & \\
            C & 이건 꼭 풀어야 해! & O & O & \\
            D & 무한부스터 & O & O & \\
            E & 우울한 방학 & O & & \\
            F & 다이나믹 롤러 & O & & \\
            G & 핑거 스냅 & O & & \\
            H & 이진수 변환 & & O & O \\
            I & 백도어 & & O & O \\
            J & FLEX & & O & \\
            K & 통신망 분할 & & O & \\
            L & 깃발춤 & & & O \\
            M & 일하는 세포 & & & O \\
            N & 트리의 외심 & & O & O \\
            O & 시간 끌기 & & & O \\
            P & 가장 높고 넓은 성 & & & O \\      
            \end{tabular}
        \end{table}
    }

    \newpage
    \import{"./problems/"}{novice-1-skh.tex}            % 와글와글 숭고한
    \import{"./problems/"}{novice-2-score.tex}          % 보너스 점수
    \import{"./problems/"}{begin-ps-must.tex}           % 이건 꼭 풀어야 해!
    \import{"./problems/"}{begin-dp-booster.tex}     % 무한부스터
    \import{"./problems/"}{free-begin-meeting.tex}              % 우울한 방학
    \import{"./problems/"}{begin-lbound-roller.tex}     % 다이나믹 롤러
    \import{"./problems/"}{begin-bfs-snap.tex}          % 핑거 스냅
    \import{"./problems/"}{free-inter-binary.tex}       % 이진수 변환
    \import{"./problems/"}{inter-dist-backdoor.tex}     % 백도어
    \import{"./problems/"}{inter-dp-flex.tex}           % FLEX
    \import{"./problems/"}{inter-disjoint-network.tex}  % 통신망 분할
    \import{"./problems/"}{adv-segtree-flag.tex}        % 깃발춤
    \import{"./problems/"}{adv-dp-cell.tex}             % 일하는 세포
    \import{"./problems/"}{inter-lca-center.tex}        % 트리의 외심
    \import{"./problems/"}{adv-flow-delay.tex}          % 시간 끌기 (수정)
    \import{"./problems/"}{adv-geo-castle.tex}          % 가장 높고 넓은 성
\end{document}
